\documentclass[]{article}
\usepackage{lmodern}
\usepackage{amssymb,amsmath}
\usepackage{ifxetex,ifluatex}
\usepackage{fixltx2e} % provides \textsubscript
\ifnum 0\ifxetex 1\fi\ifluatex 1\fi=0 % if pdftex
  \usepackage[T1]{fontenc}
  \usepackage[utf8]{inputenc}
\else % if luatex or xelatex
  \ifxetex
    \usepackage{mathspec}
  \else
    \usepackage{fontspec}
  \fi
  \defaultfontfeatures{Ligatures=TeX,Scale=MatchLowercase}
\fi
% use upquote if available, for straight quotes in verbatim environments
\IfFileExists{upquote.sty}{\usepackage{upquote}}{}
% use microtype if available
\IfFileExists{microtype.sty}{%
\usepackage{microtype}
\UseMicrotypeSet[protrusion]{basicmath} % disable protrusion for tt fonts
}{}
\usepackage[margin = 0.5in]{geometry}
\usepackage{hyperref}
\hypersetup{unicode=true,
            pdfauthor={Katherine M. Prioli},
            pdfborder={0 0 0},
            breaklinks=true}
\urlstyle{same}  % don't use monospace font for urls
\usepackage{longtable,booktabs}
\usepackage{graphicx,grffile}
\makeatletter
\def\maxwidth{\ifdim\Gin@nat@width>\linewidth\linewidth\else\Gin@nat@width\fi}
\def\maxheight{\ifdim\Gin@nat@height>\textheight\textheight\else\Gin@nat@height\fi}
\makeatother
% Scale images if necessary, so that they will not overflow the page
% margins by default, and it is still possible to overwrite the defaults
% using explicit options in \includegraphics[width, height, ...]{}
\setkeys{Gin}{width=\maxwidth,height=\maxheight,keepaspectratio}
\IfFileExists{parskip.sty}{%
\usepackage{parskip}
}{% else
\setlength{\parindent}{0pt}
\setlength{\parskip}{6pt plus 2pt minus 1pt}
}
\setlength{\emergencystretch}{3em}  % prevent overfull lines
\providecommand{\tightlist}{%
  \setlength{\itemsep}{0pt}\setlength{\parskip}{0pt}}
\setcounter{secnumdepth}{0}
% Redefines (sub)paragraphs to behave more like sections
\ifx\paragraph\undefined\else
\let\oldparagraph\paragraph
\renewcommand{\paragraph}[1]{\oldparagraph{#1}\mbox{}}
\fi
\ifx\subparagraph\undefined\else
\let\oldsubparagraph\subparagraph
\renewcommand{\subparagraph}[1]{\oldsubparagraph{#1}\mbox{}}
\fi

%%% Use protect on footnotes to avoid problems with footnotes in titles
\let\rmarkdownfootnote\footnote%
\def\footnote{\protect\rmarkdownfootnote}

%%% Change title format to be more compact
\usepackage{titling}

% Create subtitle command for use in maketitle
\providecommand{\subtitle}[1]{
  \posttitle{
    \begin{center}\large#1\end{center}
    }
}

\setlength{\droptitle}{-2em}

  \title{Understanding Correlates of Obesity:\\
Supervised and Unsupervised Learning Approaches}
    \pretitle{\vspace{\droptitle}\centering\huge}
  \posttitle{\par}
    \author{Katherine M. Prioli}
    \preauthor{\centering\large\emph}
  \postauthor{\par}
      \predate{\centering\large\emph}
  \postdate{\par}
    \date{December 05, 2019}


\begin{document}
\maketitle
\begin{abstract}
\textbf{\emph{Background}} Text \textbf{\emph{Methods}} Text
\textbf{\emph{Results}} Text \textbf{\emph{Conclusion}} Text
\end{abstract}

\hypertarget{background}{%
\subsection{\texorpdfstring{\textbf{Background}}{Background}}\label{background}}

Overweight and obesity are growing public health concerns, with 71.6\%
of American adults meeting at least the criteria for overweight, and
39.8\% qualifying as obese per the Centers for Disease Control and
Prevention (CDC) as of 2016.{[}1{]} Defined as a body mass index (BMI)
of 30 \(kg/m^{2}\) or above, obesity is of especially high concern
because it strongly correlates with many diseases and conditions that
lead to significant morbidity and mortality, and thus to increased
healthcare utilization and decreased productivity, both at high economic
burden.{[}2-4{]}

The National Health and Nutrition Examination Survey (NHANES), a
biennial health surveillance study performed by the CDC, collects data
from a nationally representative sample of community-dwelling Americans
pertaining to demographics, health, and nutritional and exercise habits
through both self-report and physical examination.{[}5{]} NHANES has
uncovered some demographic correlates of obesity \(-\) namely, that
obesity is more prevalent among Hispanics and non-Hispanic blacks for
both adults and children, that obesity is more prevalent among women
across all racial groups, and that for all racial groups except blacks,
education level appears inversely correlated with prevalence of
obesity.{[}6,7{]} However, though much progress has been made toward
understanding the socioeconomic and behavioral drivers of obesity, much
remains unknown about the interactions of these characteristics and how
constellations thereof could be used to detect obesity at the population
level.

Reducing both the incidence and prevalence of obesity will be critical
in managing obesity-related healthcare expenditure, especially for
publicly funded programs such as Medicare and Medicaid. If
constellations of obesity correlates can be identified, tailored health
interventions that target these constellations to prevent or reduce
obesity could be developed. Using the 2015-2016 NHANES data, the
objective of this study was to apply both supervised and unsupervised
machine learning approaches to find patterns in a carefully selected set
of characteristics which are known or suspected to be correlated with
obesity.

\hypertarget{methods}{%
\subsection{\texorpdfstring{\textbf{Methods}}{Methods}}\label{methods}}

\hypertarget{variable-selection}{%
\subsubsection{\texorpdfstring{\emph{Variable
Selection}}{Variable Selection}}\label{variable-selection}}

A literature search of PubMed for currently known or suspected
demographic, behavioral, and medical correlates of excess adiposity was
used to inform variable selection. NHANES 2015-2016 variables chosen for
the study are presented in Table 1, with the source denoted as
``native'' for variables native to the NHANES dataset or ``derived'' for
variables calculated from native NHANES variables. Derived variables of
note included the scored and categorized nine-item Patient Health
Questionnaire (PHQ-9), a validated nine-item depression inventory, as
well as \texttt{BP\_cat}, representing clinical categories of blood
pressure ranging from hypotension through hypertensive crisis, assigned
based on systolic and diastolic blood pressure values. Additionally,
many of the native categorical variables were refactored to group
nonresponses (e.g., refusals to respond, ``Don't know'' responses, and
missing values) into a single category.

\textbf{Table 1. Variables in initial dataset.}

\begin{longtable}[]{@{}llll@{}}
\toprule
\begin{minipage}[b]{0.10\columnwidth}\raggedright
Variable\strut
\end{minipage} & \begin{minipage}[b]{0.05\columnwidth}\raggedright
Source\strut
\end{minipage} & \begin{minipage}[b]{0.43\columnwidth}\raggedright
Description\strut
\end{minipage} & \begin{minipage}[b]{0.31\columnwidth}\raggedright
Reference\strut
\end{minipage}\tabularnewline
\midrule
\endhead
\begin{minipage}[t]{0.10\columnwidth}\raggedright
SEQN\strut
\end{minipage} & \begin{minipage}[t]{0.05\columnwidth}\raggedright
native\strut
\end{minipage} & \begin{minipage}[t]{0.43\columnwidth}\raggedright
Unique identifier\strut
\end{minipage} & \begin{minipage}[t]{0.31\columnwidth}\raggedright
{[}8{]}\strut
\end{minipage}\tabularnewline
\begin{minipage}[t]{0.10\columnwidth}\raggedright
RIAGENDR\strut
\end{minipage} & \begin{minipage}[t]{0.05\columnwidth}\raggedright
native\strut
\end{minipage} & \begin{minipage}[t]{0.43\columnwidth}\raggedright
Gender\strut
\end{minipage} & \begin{minipage}[t]{0.31\columnwidth}\raggedright
{[}8{]}\strut
\end{minipage}\tabularnewline
\begin{minipage}[t]{0.10\columnwidth}\raggedright
RIDAGEYR\strut
\end{minipage} & \begin{minipage}[t]{0.05\columnwidth}\raggedright
native\strut
\end{minipage} & \begin{minipage}[t]{0.43\columnwidth}\raggedright
Age\strut
\end{minipage} & \begin{minipage}[t]{0.31\columnwidth}\raggedright
{[}8{]}\strut
\end{minipage}\tabularnewline
\begin{minipage}[t]{0.10\columnwidth}\raggedright
RIDRETH3\strut
\end{minipage} & \begin{minipage}[t]{0.05\columnwidth}\raggedright
native\strut
\end{minipage} & \begin{minipage}[t]{0.43\columnwidth}\raggedright
Race/ethnicity\strut
\end{minipage} & \begin{minipage}[t]{0.31\columnwidth}\raggedright
{[}8{]}\strut
\end{minipage}\tabularnewline
\begin{minipage}[t]{0.10\columnwidth}\raggedright
DMDEDUC2\strut
\end{minipage} & \begin{minipage}[t]{0.05\columnwidth}\raggedright
native\strut
\end{minipage} & \begin{minipage}[t]{0.43\columnwidth}\raggedright
Education level\strut
\end{minipage} & \begin{minipage}[t]{0.31\columnwidth}\raggedright
{[}8{]}\strut
\end{minipage}\tabularnewline
\begin{minipage}[t]{0.10\columnwidth}\raggedright
DMDMARTL\strut
\end{minipage} & \begin{minipage}[t]{0.05\columnwidth}\raggedright
native\strut
\end{minipage} & \begin{minipage}[t]{0.43\columnwidth}\raggedright
Marital status\strut
\end{minipage} & \begin{minipage}[t]{0.31\columnwidth}\raggedright
{[}8{]}\strut
\end{minipage}\tabularnewline
\begin{minipage}[t]{0.10\columnwidth}\raggedright
INDFMIN2\strut
\end{minipage} & \begin{minipage}[t]{0.05\columnwidth}\raggedright
native\strut
\end{minipage} & \begin{minipage}[t]{0.43\columnwidth}\raggedright
Annual family income\strut
\end{minipage} & \begin{minipage}[t]{0.31\columnwidth}\raggedright
{[}8{]}\strut
\end{minipage}\tabularnewline
\begin{minipage}[t]{0.10\columnwidth}\raggedright
INDFMPIR\strut
\end{minipage} & \begin{minipage}[t]{0.05\columnwidth}\raggedright
native\strut
\end{minipage} & \begin{minipage}[t]{0.43\columnwidth}\raggedright
Ratio of family income to poverty\strut
\end{minipage} & \begin{minipage}[t]{0.31\columnwidth}\raggedright
{[}8{]}\strut
\end{minipage}\tabularnewline
\begin{minipage}[t]{0.10\columnwidth}\raggedright
DIQ010\strut
\end{minipage} & \begin{minipage}[t]{0.05\columnwidth}\raggedright
native\strut
\end{minipage} & \begin{minipage}[t]{0.43\columnwidth}\raggedright
History of diabetes\strut
\end{minipage} & \begin{minipage}[t]{0.31\columnwidth}\raggedright
{[}9{]}\strut
\end{minipage}\tabularnewline
\begin{minipage}[t]{0.10\columnwidth}\raggedright
DIQ280\strut
\end{minipage} & \begin{minipage}[t]{0.05\columnwidth}\raggedright
native\strut
\end{minipage} & \begin{minipage}[t]{0.43\columnwidth}\raggedright
Last hemoglobin A1C (HbA1C) level\strut
\end{minipage} & \begin{minipage}[t]{0.31\columnwidth}\raggedright
{[}9{]}\strut
\end{minipage}\tabularnewline
\begin{minipage}[t]{0.10\columnwidth}\raggedright
MCQ160c\strut
\end{minipage} & \begin{minipage}[t]{0.05\columnwidth}\raggedright
native\strut
\end{minipage} & \begin{minipage}[t]{0.43\columnwidth}\raggedright
History of coronary heart disease\strut
\end{minipage} & \begin{minipage}[t]{0.31\columnwidth}\raggedright
{[}10{]}\strut
\end{minipage}\tabularnewline
\begin{minipage}[t]{0.10\columnwidth}\raggedright
MCQ160e\strut
\end{minipage} & \begin{minipage}[t]{0.05\columnwidth}\raggedright
native\strut
\end{minipage} & \begin{minipage}[t]{0.43\columnwidth}\raggedright
History of myocardial infarction\strut
\end{minipage} & \begin{minipage}[t]{0.31\columnwidth}\raggedright
{[}10{]}\strut
\end{minipage}\tabularnewline
\begin{minipage}[t]{0.10\columnwidth}\raggedright
MCQ160m\strut
\end{minipage} & \begin{minipage}[t]{0.05\columnwidth}\raggedright
native\strut
\end{minipage} & \begin{minipage}[t]{0.43\columnwidth}\raggedright
History of thyroid disease\strut
\end{minipage} & \begin{minipage}[t]{0.31\columnwidth}\raggedright
{[}10{]}\strut
\end{minipage}\tabularnewline
\begin{minipage}[t]{0.10\columnwidth}\raggedright
MCQ365a\strut
\end{minipage} & \begin{minipage}[t]{0.05\columnwidth}\raggedright
native\strut
\end{minipage} & \begin{minipage}[t]{0.43\columnwidth}\raggedright
Doctor has told to lose weight\strut
\end{minipage} & \begin{minipage}[t]{0.31\columnwidth}\raggedright
{[}10{]}\strut
\end{minipage}\tabularnewline
\begin{minipage}[t]{0.10\columnwidth}\raggedright
MCQ365b\strut
\end{minipage} & \begin{minipage}[t]{0.05\columnwidth}\raggedright
native\strut
\end{minipage} & \begin{minipage}[t]{0.43\columnwidth}\raggedright
Doctor has told to exercise\strut
\end{minipage} & \begin{minipage}[t]{0.31\columnwidth}\raggedright
{[}10{]}\strut
\end{minipage}\tabularnewline
\begin{minipage}[t]{0.10\columnwidth}\raggedright
BPXSY2\strut
\end{minipage} & \begin{minipage}[t]{0.05\columnwidth}\raggedright
native\strut
\end{minipage} & \begin{minipage}[t]{0.43\columnwidth}\raggedright
Systolic blood pressure (\(mmHg\))\strut
\end{minipage} & \begin{minipage}[t]{0.31\columnwidth}\raggedright
{[}11{]}\strut
\end{minipage}\tabularnewline
\begin{minipage}[t]{0.10\columnwidth}\raggedright
BPXDI2\strut
\end{minipage} & \begin{minipage}[t]{0.05\columnwidth}\raggedright
native\strut
\end{minipage} & \begin{minipage}[t]{0.43\columnwidth}\raggedright
Diastolic blood pressure (\(mmHg\))\strut
\end{minipage} & \begin{minipage}[t]{0.31\columnwidth}\raggedright
{[}11{]}\strut
\end{minipage}\tabularnewline
\begin{minipage}[t]{0.10\columnwidth}\raggedright
BP\_cat\strut
\end{minipage} & \begin{minipage}[t]{0.05\columnwidth}\raggedright
derived\strut
\end{minipage} & \begin{minipage}[t]{0.43\columnwidth}\raggedright
Clinical blood pressure category\strut
\end{minipage} & \begin{minipage}[t]{0.31\columnwidth}\raggedright
{[}18, 21{]}; native variables BPXSY2, BPXDI2\strut
\end{minipage}\tabularnewline
\begin{minipage}[t]{0.10\columnwidth}\raggedright
PAD615\strut
\end{minipage} & \begin{minipage}[t]{0.05\columnwidth}\raggedright
native\strut
\end{minipage} & \begin{minipage}[t]{0.43\columnwidth}\raggedright
Average minutes of vigorous physical work per day\strut
\end{minipage} & \begin{minipage}[t]{0.31\columnwidth}\raggedright
{[}12{]}\strut
\end{minipage}\tabularnewline
\begin{minipage}[t]{0.10\columnwidth}\raggedright
PAD630\strut
\end{minipage} & \begin{minipage}[t]{0.05\columnwidth}\raggedright
native\strut
\end{minipage} & \begin{minipage}[t]{0.43\columnwidth}\raggedright
Average minutes of moderate physical work per day\strut
\end{minipage} & \begin{minipage}[t]{0.31\columnwidth}\raggedright
{[}12{]}\strut
\end{minipage}\tabularnewline
\begin{minipage}[t]{0.10\columnwidth}\raggedright
PAD660\strut
\end{minipage} & \begin{minipage}[t]{0.05\columnwidth}\raggedright
native\strut
\end{minipage} & \begin{minipage}[t]{0.43\columnwidth}\raggedright
Average minutes of vigorous physical recreational activity per day\strut
\end{minipage} & \begin{minipage}[t]{0.31\columnwidth}\raggedright
{[}12{]}\strut
\end{minipage}\tabularnewline
\begin{minipage}[t]{0.10\columnwidth}\raggedright
PAD675\strut
\end{minipage} & \begin{minipage}[t]{0.05\columnwidth}\raggedright
native\strut
\end{minipage} & \begin{minipage}[t]{0.43\columnwidth}\raggedright
Average minutes of moderate physical recreational activity per day\strut
\end{minipage} & \begin{minipage}[t]{0.31\columnwidth}\raggedright
{[}12{]}\strut
\end{minipage}\tabularnewline
\begin{minipage}[t]{0.10\columnwidth}\raggedright
mins\_activ\strut
\end{minipage} & \begin{minipage}[t]{0.05\columnwidth}\raggedright
derived\strut
\end{minipage} & \begin{minipage}[t]{0.43\columnwidth}\raggedright
Average minutes of moderate and/or vigorous activity per day\strut
\end{minipage} & \begin{minipage}[t]{0.31\columnwidth}\raggedright
Native variables PAD615, PAD630, PAD660, PAD675\strut
\end{minipage}\tabularnewline
\begin{minipage}[t]{0.10\columnwidth}\raggedright
PAD680\strut
\end{minipage} & \begin{minipage}[t]{0.05\columnwidth}\raggedright
native\strut
\end{minipage} & \begin{minipage}[t]{0.43\columnwidth}\raggedright
Average minutes awake and sedentary per day\strut
\end{minipage} & \begin{minipage}[t]{0.31\columnwidth}\raggedright
{[}12{]}\strut
\end{minipage}\tabularnewline
\begin{minipage}[t]{0.10\columnwidth}\raggedright
PFQ049\strut
\end{minipage} & \begin{minipage}[t]{0.05\columnwidth}\raggedright
native\strut
\end{minipage} & \begin{minipage}[t]{0.43\columnwidth}\raggedright
Unable to work due to impairment\strut
\end{minipage} & \begin{minipage}[t]{0.31\columnwidth}\raggedright
{[}13{]}\strut
\end{minipage}\tabularnewline
\begin{minipage}[t]{0.10\columnwidth}\raggedright
PFQ061\strut
\end{minipage} & \begin{minipage}[t]{0.05\columnwidth}\raggedright
native\strut
\end{minipage} & \begin{minipage}[t]{0.43\columnwidth}\raggedright
Difficulty walking 1/4 mile\strut
\end{minipage} & \begin{minipage}[t]{0.31\columnwidth}\raggedright
{[}13{]}\strut
\end{minipage}\tabularnewline
\begin{minipage}[t]{0.10\columnwidth}\raggedright
DPQ010 - DPQ090\strut
\end{minipage} & \begin{minipage}[t]{0.05\columnwidth}\raggedright
native\strut
\end{minipage} & \begin{minipage}[t]{0.43\columnwidth}\raggedright
Depression inventory subscores\strut
\end{minipage} & \begin{minipage}[t]{0.31\columnwidth}\raggedright
{[}14{]}\strut
\end{minipage}\tabularnewline
\begin{minipage}[t]{0.10\columnwidth}\raggedright
PHQ9\_score\strut
\end{minipage} & \begin{minipage}[t]{0.05\columnwidth}\raggedright
derived\strut
\end{minipage} & \begin{minipage}[t]{0.43\columnwidth}\raggedright
Depression inventory score\strut
\end{minipage} & \begin{minipage}[t]{0.31\columnwidth}\raggedright
{[}19{]}; native variables DPQ010 - DPQ090\strut
\end{minipage}\tabularnewline
\begin{minipage}[t]{0.10\columnwidth}\raggedright
PHQ9\_cat\strut
\end{minipage} & \begin{minipage}[t]{0.05\columnwidth}\raggedright
derived\strut
\end{minipage} & \begin{minipage}[t]{0.43\columnwidth}\raggedright
Depression inventory category\strut
\end{minipage} & \begin{minipage}[t]{0.31\columnwidth}\raggedright
{[}19{]}; native variables DPQ010 - DPQ090\strut
\end{minipage}\tabularnewline
\begin{minipage}[t]{0.10\columnwidth}\raggedright
DBQ700\strut
\end{minipage} & \begin{minipage}[t]{0.05\columnwidth}\raggedright
native\strut
\end{minipage} & \begin{minipage}[t]{0.43\columnwidth}\raggedright
Self-perception of dietary healthiness\strut
\end{minipage} & \begin{minipage}[t]{0.31\columnwidth}\raggedright
{[}15{]}\strut
\end{minipage}\tabularnewline
\begin{minipage}[t]{0.10\columnwidth}\raggedright
CBQ505\strut
\end{minipage} & \begin{minipage}[t]{0.05\columnwidth}\raggedright
native\strut
\end{minipage} & \begin{minipage}[t]{0.43\columnwidth}\raggedright
Ate fast food or pizza within past 12 months\strut
\end{minipage} & \begin{minipage}[t]{0.31\columnwidth}\raggedright
{[}15{]}\strut
\end{minipage}\tabularnewline
\begin{minipage}[t]{0.10\columnwidth}\raggedright
CBQ540\strut
\end{minipage} & \begin{minipage}[t]{0.05\columnwidth}\raggedright
native\strut
\end{minipage} & \begin{minipage}[t]{0.43\columnwidth}\raggedright
Used nutritional information to choose fast foods\strut
\end{minipage} & \begin{minipage}[t]{0.31\columnwidth}\raggedright
{[}15{]}\strut
\end{minipage}\tabularnewline
\begin{minipage}[t]{0.10\columnwidth}\raggedright
CBQ545\strut
\end{minipage} & \begin{minipage}[t]{0.05\columnwidth}\raggedright
native\strut
\end{minipage} & \begin{minipage}[t]{0.43\columnwidth}\raggedright
Would use nutritional information to choose fast foods\strut
\end{minipage} & \begin{minipage}[t]{0.31\columnwidth}\raggedright
{[}15{]}\strut
\end{minipage}\tabularnewline
\begin{minipage}[t]{0.10\columnwidth}\raggedright
CBQ550\strut
\end{minipage} & \begin{minipage}[t]{0.05\columnwidth}\raggedright
native\strut
\end{minipage} & \begin{minipage}[t]{0.43\columnwidth}\raggedright
Ate at a restaurant with waitstaff in past 12 months\strut
\end{minipage} & \begin{minipage}[t]{0.31\columnwidth}\raggedright
{[}15{]}\strut
\end{minipage}\tabularnewline
\begin{minipage}[t]{0.10\columnwidth}\raggedright
CBQ585\strut
\end{minipage} & \begin{minipage}[t]{0.05\columnwidth}\raggedright
native\strut
\end{minipage} & \begin{minipage}[t]{0.43\columnwidth}\raggedright
Used nutritional information to choose restaurant meal\strut
\end{minipage} & \begin{minipage}[t]{0.31\columnwidth}\raggedright
{[}15{]}\strut
\end{minipage}\tabularnewline
\begin{minipage}[t]{0.10\columnwidth}\raggedright
CBQ590\strut
\end{minipage} & \begin{minipage}[t]{0.05\columnwidth}\raggedright
native\strut
\end{minipage} & \begin{minipage}[t]{0.43\columnwidth}\raggedright
Would use nutritional information to choose restaurant meal\strut
\end{minipage} & \begin{minipage}[t]{0.31\columnwidth}\raggedright
{[}15{]}\strut
\end{minipage}\tabularnewline
\begin{minipage}[t]{0.10\columnwidth}\raggedright
DR1TKCAL\strut
\end{minipage} & \begin{minipage}[t]{0.05\columnwidth}\raggedright
native\strut
\end{minipage} & \begin{minipage}[t]{0.43\columnwidth}\raggedright
Dietary intake, Day 1 (\emph{kcal})\strut
\end{minipage} & \begin{minipage}[t]{0.31\columnwidth}\raggedright
{[}16{]}\strut
\end{minipage}\tabularnewline
\begin{minipage}[t]{0.10\columnwidth}\raggedright
DR1300\strut
\end{minipage} & \begin{minipage}[t]{0.05\columnwidth}\raggedright
native\strut
\end{minipage} & \begin{minipage}[t]{0.43\columnwidth}\raggedright
Day 1 dietary intake compared to usual\strut
\end{minipage} & \begin{minipage}[t]{0.31\columnwidth}\raggedright
{[}16{]}\strut
\end{minipage}\tabularnewline
\begin{minipage}[t]{0.10\columnwidth}\raggedright
DR1\_32OZ\strut
\end{minipage} & \begin{minipage}[t]{0.05\columnwidth}\raggedright
native\strut
\end{minipage} & \begin{minipage}[t]{0.43\columnwidth}\raggedright
Water intake, Day 1 (\emph{g})\strut
\end{minipage} & \begin{minipage}[t]{0.31\columnwidth}\raggedright
{[}16{]}\strut
\end{minipage}\tabularnewline
\begin{minipage}[t]{0.10\columnwidth}\raggedright
BMXBMI\strut
\end{minipage} & \begin{minipage}[t]{0.05\columnwidth}\raggedright
native\strut
\end{minipage} & \begin{minipage}[t]{0.43\columnwidth}\raggedright
Body Mass Index (\(kg/m^{2}\))\strut
\end{minipage} & \begin{minipage}[t]{0.31\columnwidth}\raggedright
{[}17{]}\strut
\end{minipage}\tabularnewline
\begin{minipage}[t]{0.10\columnwidth}\raggedright
BMI\_cat\strut
\end{minipage} & \begin{minipage}[t]{0.05\columnwidth}\raggedright
derived\strut
\end{minipage} & \begin{minipage}[t]{0.43\columnwidth}\raggedright
BMI weight category\strut
\end{minipage} & \begin{minipage}[t]{0.31\columnwidth}\raggedright
{[}2{]}; native variable BMXBMI\strut
\end{minipage}\tabularnewline
\bottomrule
\end{longtable}

Because BMI categories differ for children and adults, the analysis was
limited to ages \(\ge\) 20, consistent with the NHANES definition of
adult.{[}8{]} Additionally, since \texttt{BMI\_cat} represents the data
labels, all rows with null \texttt{BMI\_cat} were excluded.

Descriptive statistics and frequency tables were generated for
continuous and categorical variables respectively to understand data
contents and to assess the degree of missingness among the continuous
data. Missing values were imputed for continuous data using the R
function \texttt{rfImpute()}. To avoid introducing bias through
imputation, two-sided Wilcoxon Rank-Sum tests were performed for each
variable on the pre- vs.~post-imputation data to identify any
statistically significant changes introduced by imputation at the
\(\alpha = 0.05\) level. Any variables having statistically different
post-imputation data were removed from the dataset.

\hypertarget{supervised-approach}{%
\subsubsection{\texorpdfstring{\emph{Supervised
Approach}}{Supervised Approach}}\label{supervised-approach}}

Using algorithms available in Scikit-learn, a random forest of decision
tree classifiers was used with \(n \times k\)-fold crossvalidation to
classify cases into BMI category, and descriptive statistics were
generated to assess model accuracy.{[}22-24{]} To improve performance,
hyperparameters \texttt{n\_repeats} (or \emph{n}, number of times
crossvalidation was performed) and \texttt{n\_splits} (or \emph{k},
number of folds) in \texttt{RepeatedKFold()} were tuned, along with
hyperparameters \texttt{n\_estimators} (number of decision trees in the
forest), \texttt{min\_samples\_leaf} (minimum cases allowed per terminal
node), and \texttt{max\_depth} (maximum tree depth) in
\texttt{RandomForestClassifier}.

To further improve accuracy, the labels representing BMI category were
recast as dichotomous, grouping together the underweight and healthy
weight categories, versus the overweight and obese categories.
Additionally, the six comorbidity variables (pertaining to diabetes,
coronary artery disease, myocardial infarction, thyroid disease,
hypertension, and depression) were consolidated into one variable
representing number of comorbidities, and variables which were suspected
not to contribute much information to the model (due to low variation or
large amount of missingness) were excluded. This modified dataset was
then run through \(N \times k\)-fold crossvalidation with hyperparameter
tuning and model accuracy assessment as previously described.

\hypertarget{unsupervised-approach}{%
\subsubsection{\texorpdfstring{\emph{Unsupervised
Approach}}{Unsupervised Approach}}\label{unsupervised-approach}}

For the unsupervised component of this analysis, data was subset to
variables with \(\ge 90\)\% non-missing values, then all cases with
missing values were dropped. Since this approach involves visual
analysis via dendrograms, the data was further subset to a random 5\%
sample within each BMI category to allow for sensible, interpretable
dendrogram plots. The analytic dataset thus comprised 241 cases having
17 variables. To generate dendrograms for agglomerative clustering
(AGNES) and divisive clustering (DIANA), a dissimilarity matrix was
calculated based on Gower distance.{[}28{]} The Gower distance was
chosen because the variables in this dataset are of mixed type (i.e.,
some continuous, some ordinal, and some nominal). Because the data is
likely noisy, AGNES relied on complete (i.e., maximum distance) linkage
between clusters.

Radial dendrograms for both AGNES and DIANA were plotted along with
colored labels at each leaf to indicate the four BMI weight categories.
For both AGNES and DIANA, cluster size was assessed and within-cluster
sums of squares and average silhouette width were calculated based on an
upper limit of 15 clusters; these latter two metrics were used to
generate scree and silhouette plots, which were used to determine an
initial cluster number for each approach. The radial dendrograms were
replotted using these values to inform cluster coloration. Homogeneity
of label colors in each cluster was assessed via visual inspection, and
the cluster number hyperparameter \emph{k} was tuned as needed to
improve homogeneity.

\hypertarget{technologies}{%
\subsubsection{\texorpdfstring{\emph{Technologies}}{Technologies}}\label{technologies}}

Data import, wrangling, and visualization were performed in R using an
RMarkdown notebook and relying primarily on the \texttt{tidyverse}
ecosystem. The \(n \times k\) crossvalidation random forest was
implemented via Python code chunks in the notebook leveraging
Scikit-Learn along with other common Python libraries as necessary
(\emph{e.g.}, \texttt{numpy} and \texttt{pandas}).{[}22-24{]} For the
agglomerative and divisive clustering, R packages \texttt{cluster},
\texttt{fpc}, \texttt{dendrogram} and \texttt{dendextend} were used. A
public GitHub repository was established for this study and contains all
raw data files and code along with project documentation.{[}20{]}

\hypertarget{results}{%
\subsection{\texorpdfstring{\textbf{Results}}{Results}}\label{results}}

The final analytic dataframe contained 5406 rows and 24 columns.
Variables included in the dataset are presented in Table 2:

\textbf{Table 2. Variables in final dataset.}

\begin{longtable}[]{@{}ll@{}}
\toprule
\begin{minipage}[b]{0.24\columnwidth}\raggedright
Variable\strut
\end{minipage} & \begin{minipage}[b]{0.70\columnwidth}\raggedright
Description\strut
\end{minipage}\tabularnewline
\midrule
\endhead
\begin{minipage}[t]{0.24\columnwidth}\raggedright
age\strut
\end{minipage} & \begin{minipage}[t]{0.70\columnwidth}\raggedright
Age\strut
\end{minipage}\tabularnewline
\begin{minipage}[t]{0.24\columnwidth}\raggedright
BMI\_cat\strut
\end{minipage} & \begin{minipage}[t]{0.70\columnwidth}\raggedright
BMI weight category\strut
\end{minipage}\tabularnewline
\begin{minipage}[t]{0.24\columnwidth}\raggedright
dailykcal\strut
\end{minipage} & \begin{minipage}[t]{0.70\columnwidth}\raggedright
Calories consumed on previous day\strut
\end{minipage}\tabularnewline
\begin{minipage}[t]{0.24\columnwidth}\raggedright
dailykcal\_typical\strut
\end{minipage} & \begin{minipage}[t]{0.70\columnwidth}\raggedright
Previous day's calorie consumption compared to usual\strut
\end{minipage}\tabularnewline
\begin{minipage}[t]{0.24\columnwidth}\raggedright
dailywater\strut
\end{minipage} & \begin{minipage}[t]{0.70\columnwidth}\raggedright
Previous day's water intake\strut
\end{minipage}\tabularnewline
\begin{minipage}[t]{0.24\columnwidth}\raggedright
diethealthy\strut
\end{minipage} & \begin{minipage}[t]{0.70\columnwidth}\raggedright
Feels diet is healthy\strut
\end{minipage}\tabularnewline
\begin{minipage}[t]{0.24\columnwidth}\raggedright
educ\strut
\end{minipage} & \begin{minipage}[t]{0.70\columnwidth}\raggedright
Level of education\strut
\end{minipage}\tabularnewline
\begin{minipage}[t]{0.24\columnwidth}\raggedright
famincome\_cat\strut
\end{minipage} & \begin{minipage}[t]{0.70\columnwidth}\raggedright
Family income category\strut
\end{minipage}\tabularnewline
\begin{minipage}[t]{0.24\columnwidth}\raggedright
fastfood\_eat\strut
\end{minipage} & \begin{minipage}[t]{0.70\columnwidth}\raggedright
Has eaten fast food in past 12 months\strut
\end{minipage}\tabularnewline
\begin{minipage}[t]{0.24\columnwidth}\raggedright
fastfood\_usednutrit\strut
\end{minipage} & \begin{minipage}[t]{0.70\columnwidth}\raggedright
Has used nutrition information to select fast food\strut
\end{minipage}\tabularnewline
\begin{minipage}[t]{0.24\columnwidth}\raggedright
fastfood\_woulduse\strut
\end{minipage} & \begin{minipage}[t]{0.70\columnwidth}\raggedright
Would use nutrition information to select fast food\strut
\end{minipage}\tabularnewline
\begin{minipage}[t]{0.24\columnwidth}\raggedright
gender\strut
\end{minipage} & \begin{minipage}[t]{0.70\columnwidth}\raggedright
Gender\strut
\end{minipage}\tabularnewline
\begin{minipage}[t]{0.24\columnwidth}\raggedright
losewt\_exer\strut
\end{minipage} & \begin{minipage}[t]{0.70\columnwidth}\raggedright
Told by doctor to lose weight and/or exercise\strut
\end{minipage}\tabularnewline
\begin{minipage}[t]{0.24\columnwidth}\raggedright
marital\strut
\end{minipage} & \begin{minipage}[t]{0.70\columnwidth}\raggedright
Marital status\strut
\end{minipage}\tabularnewline
\begin{minipage}[t]{0.24\columnwidth}\raggedright
mins\_activ\strut
\end{minipage} & \begin{minipage}[t]{0.70\columnwidth}\raggedright
Daily minutes of moderate to vigorous activity\strut
\end{minipage}\tabularnewline
\begin{minipage}[t]{0.24\columnwidth}\raggedright
mins\_seden\strut
\end{minipage} & \begin{minipage}[t]{0.70\columnwidth}\raggedright
Daily minutes spent sedentary\strut
\end{minipage}\tabularnewline
\begin{minipage}[t]{0.24\columnwidth}\raggedright
n\_comorbid\strut
\end{minipage} & \begin{minipage}[t]{0.70\columnwidth}\raggedright
Number of comorbidities\strut
\end{minipage}\tabularnewline
\begin{minipage}[t]{0.24\columnwidth}\raggedright
race\strut
\end{minipage} & \begin{minipage}[t]{0.70\columnwidth}\raggedright
Race\strut
\end{minipage}\tabularnewline
\begin{minipage}[t]{0.24\columnwidth}\raggedright
restaur\_eat\strut
\end{minipage} & \begin{minipage}[t]{0.70\columnwidth}\raggedright
Has eaten at a restaurant in past 12 months\strut
\end{minipage}\tabularnewline
\begin{minipage}[t]{0.24\columnwidth}\raggedright
restaur\_usednutrit\strut
\end{minipage} & \begin{minipage}[t]{0.70\columnwidth}\raggedright
Has used nutrition information to select restaurant meal\strut
\end{minipage}\tabularnewline
\begin{minipage}[t]{0.24\columnwidth}\raggedright
restaur\_woulduse\strut
\end{minipage} & \begin{minipage}[t]{0.70\columnwidth}\raggedright
Would use nutrition information to select restaurant meal\strut
\end{minipage}\tabularnewline
\begin{minipage}[t]{0.24\columnwidth}\raggedright
seqn\strut
\end{minipage} & \begin{minipage}[t]{0.70\columnwidth}\raggedright
Unique identifier (omitted from analyses)\strut
\end{minipage}\tabularnewline
\begin{minipage}[t]{0.24\columnwidth}\raggedright
walklim\strut
\end{minipage} & \begin{minipage}[t]{0.70\columnwidth}\raggedright
Walking limitations\strut
\end{minipage}\tabularnewline
\begin{minipage}[t]{0.24\columnwidth}\raggedright
worklim\strut
\end{minipage} & \begin{minipage}[t]{0.70\columnwidth}\raggedright
Work limitations\strut
\end{minipage}\tabularnewline
\bottomrule
\end{longtable}

\hypertarget{supervised-approach-1}{%
\subsubsection{\texorpdfstring{\emph{Supervised
Approach}}{Supervised Approach}}\label{supervised-approach-1}}

The initial model, based on the full dataset and using \(k = 5\) folds,
\(n = 5\) repeats, and 100 trees in the ensemble without limitation on
the number of cases per terminal node, yielded 52\% mean accuracy
(range: 49-55\%). Variables included in the reduced dataset included
age, gender, race, education level, marital status, family income
category, number of comorbidities, number of daily minutes active and
sedentary, having eaten fast food or at a restaurant in the past year,
daily calories, daily water intake, and having been told by a doctor to
lose weight and/or exercise, along with dichotomous BMI category. The
final model based on this reduced dataset used \(k = 3\) folds,
\(n = 10\) repeats, and 200 trees, and was limited to \(\ge 25\) cases
per terminal node. This model yielded 75.53\% mean accuracy (range:
73.58-77.03\%), representing a meaningful improvement.

\hypertarget{unsupervised-approach-1}{%
\subsubsection{\texorpdfstring{\emph{Unsupervised
Approach}}{Unsupervised Approach}}\label{unsupervised-approach-1}}

The scree and silhouette plots are presented in Fig. 1. The AGNES scree
plot shows a relatively smooth curve without strong elbows; possible but
faint elbows appear at 5 and 7 clusters. For DIANA, two elbows are seen:
one at 5 clusters and one at 6. Both the AGNES and DIANA silhouette
plots show maximum average silhouette width at 2 clusters, which is
unlikely to contain sufficient data to be meaningful. AGNES shows a
local maximum at 7 clusters, and DIANA at 6. Considered together, scree
and silhouette plots indicate that 7 clusters may be sufficient for
AGNES, and 6 for DIANA.

\textbf{Figure 3. Scree and Silhouette Plots}

\includegraphics{Prioli_final_report_files/figure-latex/scree_silhou-1.pdf}
\includegraphics{Prioli_final_report_files/figure-latex/scree_silhou-2.pdf}

Initial radial dendrograms are shown in Figs. 2 and 3.

\textbf{Figure 3. Initial Colored Radial Dendrogram, AGNES (\emph{k} =
7)}

\includegraphics{Prioli_final_report_files/figure-latex/agnes_dend_clust_init-1.pdf}

\textbf{Figure 3. Initial Colored Radial Dendrogram, DIANA (\emph{k} =
6)}

\includegraphics{Prioli_final_report_files/figure-latex/diana_dend_clust_init-1.pdf}

\hypertarget{discussion}{%
\subsection{\texorpdfstring{\textbf{Discussion}}{Discussion}}\label{discussion}}

For the supervised approach, the final decision model reached 75.53\%
mean accuracy. While this is an improvement over the initial model, it
generally would be considered poor performance for any real-world
implementation.

\hypertarget{strengths}{%
\subsubsection{\texorpdfstring{\emph{Strengths}}{Strengths}}\label{strengths}}

\hypertarget{limitations}{%
\subsubsection{\texorpdfstring{\emph{Limitations}}{Limitations}}\label{limitations}}

Because the Scikit-Learn \texttt{randomForestClassifier()} function was
unable to handle missing data, missing values were handled via
imputation (continuous variables) or by creating a ``missing'' category
comprising all-cause nonresponses (categorical variables). Using a
``missing'' category in this manner may have introduced noise into the
dataset, because it's not known missing values were missing completely
at random or missing not at random. Future iterations of this analysis
should include other random forest functions - e.g., those available in
the \texttt{xgboost} Python library.

Another limitation is that variable selection is difficult with
continuous variables. Principal Component Analysis (PCA) is a common
method used for choosing variables when considering continuous data, but
is not commonly used when data is categorical. There is, however,
evidence in the literature to suggest that categorical variables can be
treated via a modified PCA approach, but the packages currently
available for this either have limited documentation or are not built on
a recent version of R. {[}25-27{]} In the absence of more sophisticated
tools, variable selection was based on those characteristics expected to
be most closely correlated with weight.

\hypertarget{conclusion}{%
\subsection{\texorpdfstring{\textbf{Conclusion}}{Conclusion}}\label{conclusion}}

\hypertarget{references}{%
\subsection{\texorpdfstring{\textbf{References}}{References}}\label{references}}

\begin{enumerate}
\def\labelenumi{\arabic{enumi}.}
\item
  Selected health conditions and risk factors, by age: United States,
  selected years 1988-1994 through 2015-2016. Centers for Disease
  Control and Prevention. Health, United States, 2017: Trend Tables.
  \url{https://www.cdc.gov/nchs/data/hus/2017/053.pdf}. Accessed October
  20, 2019.
\item
  Defining Adult Overweight and Obesity. Centers for Disease Control and
  Prevention. \url{https://www.cdc.gov/obesity/adult/defining.html}.
  Updated April 11, 2017. Accessed October 22, 2019.
\item
  Hruby A, Hu FB. The Epidemiology of Obesity: A Big Picture.
  \emph{Pharmacoeconomics}. 2015;33(7):673-89.
  \url{doi:10.1007/s40273-014-0243-x}.
\item
  Li Q, Blume SW, Huang JC, Hammer M, Ganz ML. Prevalence and healthcare
  costs of obesity-related comorbidities: evidence from an electronic
  medical records system in the United States. \emph{J Med Econ}.
  2015;18(12):1020-8. doi: 10.3111/13696998.2015.1067623.
\item
  National Health and Nutrition Examination Survey. National Center for
  Health Statistics. \url{https://www.cdc.gov/nchs/nhanes/index.htm}.
  Updated September 24, 2019. Accessed October 20, 2019.
\item
  Hales CM, Carroll MD, Fryar CD, Ogden CL. Prevalence of Obesity Among
  Adults and Youth: United States, 2015-2016. NCHS Data Brief No.~288.
  National Center for Health Statistics. 2017.
  \url{https://www.cdc.gov/nchs/data/databriefs/db288.pdf}. Accessed
  October 20, 2019.
\item
  Ogden CL, Fakhouri TH, Carroll MD, et al.~Prevalence of Obesity Among
  Adults, by Household Income and Education -- United States, 2011-2014.
  \emph{Morb Mortal Wkly Rep}. 2017;66:1369-1373. doi:
  \url{http://dx.doi.org/10.15585/mmwr.mm6650a1}.
\item
  National Health and Nutrition Examination Survey: Demographic
  Variables and Sample Weights (DEMO\_I). National Center for Health
  Statistics.
  \url{https://wwwn.cdc.gov/Nchs/Nhanes/2015-2016/DEMO_I.XPT}. Updated
  September 2017. Accessed October 27, 2019.
\item
  National Health and Nutrition Examination Survey: Questionnaire Data -
  Diabetes (DIQ\_I). National Center for Health Statistics.
  \url{https://wwwn.cdc.gov/Nchs/Nhanes/2015-2016/DIQ_I.XPT}. Updated
  September 2017. Accessed October 29, 2019.
\item
  National Health and Nutrition Examination Survey: Questionnaire Data -
  Medical (MCQ\_I). National Center for Health Statistics.
  \url{https://wwwn.cdc.gov/Nchs/Nhanes/2015-2016/MCQ_I.XPT}. Updated
  September 2017. Accessed October 29, 2019.
\item
  National Health and Nutrition Examination Survey: Examination Data -
  Blood Pressure (BPX\_I). National Center for Health Statistics.
  \url{https://wwwn.cdc.gov/Nchs/Nhanes/2015-2016/BPX_I.XPT}. Updated
  September 2017. Accessed October 29, 2019.
\item
  National Health and Nutrition Examination Survey: Questionnaire Data -
  Physical Activity (PAQ\_I). National Center for Health Statistics.
  \url{https://wwwn.cdc.gov/Nchs/Nhanes/2015-2016/PAQ_I.XPT}. Updated
  September 2017. Accessed October 29, 2019.
\item
  National Health and Nutrition Examination Survey: Questionnaire Data -
  Physical Functioning (PFQ\_I). National Center for Health Statistics.
  \url{https://wwwn.cdc.gov/Nchs/Nhanes/2015-2016/PFQ_I.XPT}. Updated
  September 2017. Accessed October 29, 2019.
\item
  National Health and Nutrition Examination Survey: Questionnaire Data -
  Mental Health: Depression Screener (DPQ\_I). National Center for
  Health Statistics.
  \url{https://wwwn.cdc.gov/Nchs/Nhanes/2015-2016/DPQ_I.XPT}. Updated
  December 2017. Accessed October 29, 2019.
\item
  National Health and Nutrition Examination Survey: Questionnaire Data -
  Diet Behavior and Nutrition (DBQ\_I). National Center for Health
  Statistics.
  \url{https://wwwn.cdc.gov/Nchs/Nhanes/2015-2016/DBQ_I.XPT}. Updated
  November 2018. Accessed October 29, 2019.
\item
  National Health and Nutrition Examination Survey: Questionnaire Data -
  Dietary Interview: Total Nutrient Intakes, First Day (DR1TOT\_I).
  National Center for Health Statistics.
  \url{https://wwwn.cdc.gov/Nchs/Nhanes/2015-2016/DR1TOT_I.XPT}. Updated
  July 2018. Accessed October 29, 2019.
\item
  National Health and Nutrition Examination Survey: Examination Data -
  Body Measures (BMX\_I). National Center for Health Statistics.
  \url{https://wwwn.cdc.gov/Nchs/Nhanes/2015-2016/BMX_I.XPT}. Updated
  September 2017. Accessed October 29, 2019.
\item
  Whelton PK, Carey RM, Aronow WS, et al.~2017
  ACC/AHA/AAPA/ABC/ACPM/AGS/APhA/ASH/ASPC/NMA/PCNA Guideline for the
  Prevention, Detection, Evaluation, and Management of High Blood
  Pressure in Adults: A Report of the American College of
  Cardiology/American Heart Association Task Force on Clinical Practice
  Guidelines. \emph{J Am Coll Cardiol}. 2018;71(19):e127-e248. doi:
  10.1016/j.jacc.2017.11.006.
\item
  Kroenke K, Spitzer RL. The PHQ-9: A new depression diagnostic and
  severity measure. \emph{Psychiatric Annals}. 2002;32(9):1-7 doi:
  10.3928/0048-5713-20020901-06.
\item
  Prioli KM. CSC\_8515\_Final\_Project.
  \url{https://github.com/kprioliPROF/CSC_8515_Final_Project}.
\item
  Low Blood Pressure. US National Library of Medicine: MedlinePlus.
  \url{https://medlineplus.gov/ency/article/007278.htm}. Updated
  November 06, 2019. Accessed November 10, 2019.
\item
  sklearn.ensemble.RandomForestClassifier. Scikit-learn documentation.
  \url{https://scikit-learn.org/stable/modules/generated/sklearn.ensemble.RandomForestClassifier.html}.
  Accessed November 10, 2019.
\item
  sklearn.model\_selection.RepeatedKFold. Scikit-learn documentation.
  \url{https://scikit-learn.org/stable/modules/generated/sklearn.model_selection.RepeatedKFold.html}.
  Accessed November 10, 2019.
\item
  sklearn.model\_selection.cross\_val\_score. Scikit-learn
  documentation.
  \url{https://scikit-learn.org/stable/modules/generated/sklearn.model_selection.cross_val_score.html}.
  Accessed November 10, 2019.
\item
  Niitsuma H, Okada T. Covariance and PCA for Categorical Variables.
  Advances in Knowledge Discovery and Data Mining. Ho TB, Cheung D, Liu
  H, eds.~*Advances in Knowledge Discovery and Data Mining. Berlin:
  Springer; 2005. doi: 10.1007/11430919\_61. arXiv: 0711.4452
  {[}cs.LG{]}.
\item
  princals: Categorical principal component analysis (PRINCALS). Gifi
  Multivariate Analysis with Optimal Scaling documentation.
  \url{https://cran.r-project.org/web/packages/Gifi/index.html}. Updated
  June 25, 2019. Accessed November 17, 2019.
\item
  Chavent M, Kuentz-Simonet V, Labenne A, Saracco J. Multivariate
  Analysis of Mixed Data: The R Package PCAmixdata. arXiv:1411.4911v4
  {[}stat.CO{]} Updated December 8, 2017. Accessed November 17, 2019.
\item
  Gower JC. A General Coefficient of Similarity and Some of Its
  Properties. \emph{Biometrics}. 1971;27(4):857-71. doi:
  10.2307/2528823.
\end{enumerate}


\end{document}
